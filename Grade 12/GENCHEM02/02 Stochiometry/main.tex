\documentclass{article}
\usepackage[T1]{fontenc}
\usepackage{amsmath}
\usepackage{chemformula}

\begin{document}

\subsubsection{Wumbo Word Wednesday}

Effervescence
\begin{itemize}
    \item the act of bubbling out
    \item the process in which gas bubbles out of a solution
    \item alt: to be enthusiastic
\end{itemize}

\subsection*{Percent Composition}
Percent Composition
\begin{itemize}
    \item a way to express the relative amounts of atoms in a molecule through mass percentages.
    \item mass / element * 100
    \item \ch{Fe2O3}
    \item Chemical Formulas with the same composition will have the same ratio
\end{itemize}
Empirical Formula
\begin{itemize}
    \item reacall that subscripts represent the ratios of each atom in the molecule
    \item think of it as simplifying algebra
    \item we must base the ratio off the element with the least number of moles
    \item Convert percent to mass; best to assume 100grams for easier conversion
    \item convert mass to moles using molar masses
    \item get the ratio of the moles
\end{itemize}
Percent Formulas
\begin{itemize}
    \item using 100grams divide it by its molar mass to find the amount of moles, and then determine its Empirical Formula
    \item if its 0.33 or 0.66 multiply bt 3
    \item Find the Empirical Formula
    \item Calculate for empricial mass and add molar masses of each element
    \item divide molar mass by empirical mass
    \item Multiply empirical formula by rounded off number
    \item Can only fe
\end{itemize}
\end{document}